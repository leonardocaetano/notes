\documentclass[a4paper, 12pt]{book}
\usepackage[a4paper, left=0.5in, right=0.5in, top=0.8in, bottom=0.8in]{geometry}

\usepackage{hyperref} % this generates the bookmarks for PDFs
\usepackage[english]{babel}
\usepackage{amsfonts} % for math sets letters
\usepackage{amsmath} % for general math stuff
\usepackage{amssymb} % for general math stuff
\usepackage{blindtext} % generates filler text
\usepackage{paralist} % for the compactenum and compactlist procedures
% ====== MY MACROS

\newcommand{\TUG}{\TeX\ Users Group}
\newcommand{\kw}[2][\bfseries]{{#1#2}}
\newcommand{\im}[1]{{\(#1\)}}
\newcommand{\cm}[1]{{\[#1\]}} % remember that when you want to do center math you don't need to have \\ on EOL
\newcommand{\head}[1]{\textnormal{\textbf{#1}}}
\newtheorem{thm}{Theorem}
\newtheorem{dfn}[thm]{Definition}
% ====== MY MACROS

\title{\LaTeX\ introduction}
\author{Me}
\date{Jan 27, 2024}

\addtocontents{toc}{\protect{\pdfbookmark[0]{\contentsname}{toc}}}

\begin{document}
\maketitle
\tableofcontents
\chapter*{Preface}
\blindtext[20]
\chapter{Introduction}
\section{Some basics about \LaTeX}
Latex is not word processing, but typesetting. That means you have to provide precise instructions with text.\\
Statement \#1:
50\% of \$100 equals to \$50.\\
More special symbols are \&, \_, \{ and \}.\\
Text can be \emph{emphasized.}\\
Besides from \textit{italics}, words can be \textbf{bold}, \textsl{slanted}, or typeset in \textsc{Small Caps}.\\
Such commands can be \textit{\textbf{nested}}.\\
\emph{See how \emph{emphasizing} looks when nested.}\\ % \(1+x^2\) \[1+x^2\] % () for inline math, [] for center math, $$ is old
\texttt{This is a monospaced font. Also called typewriter font.}\\
A curly brace means in \LaTeX\ that we about to start a group.\\ 
\tiny We \scriptsize can \footnotesize also \small do \normalsize this.\\
The \TUG\ is a result of a custom macro. Use them!\\
Macros can also receive \kw{arguments}.
\cm{1+2=3}
That makes it easier when you want to do \im{2^3=8} for example.
\newpage
\section{How to write "books"}
\blindtext[20]
And as you can see, we need a \footnote{footnote now. We really do. Remember that \im{3^2 = 9}.}
\newpage
\section{How to write lists}
\begin{enumerate}
    \item C
        \begin{itemize}
            \item C89
            \item C99
            \item C11
            \item C17
            \item C23
        \end{itemize}
    \item C++
    \item Pascal
    \item asm 
    \item Rust
\end{enumerate}
Note that \texttt{enumerate} and \texttt{itemize} has the same syntax.
\begin{description}
    \item[C89] is the best one.
    \item[C23] is a Rust pysop.
\end{description}
We can also use \texttt{compactenum} and \texttt{compactitem} if we use the package \texttt{paralist}.\\
\begin{compactenum}
    \item C
        \begin{compactitem}
            \item C89
            \item C99
            \item C11
            \item C17
            \item C23
        \end{compactitem}
    \item C++
    \item Pascal
    \item asm 
    \item Rust
\end{compactenum}
\newpage
\section{How to create tables}
This looks like a PITA. It probably is.
\begin{tabbing}
    \emph{Info:} \= Software \= :  \= \LaTeX \\
    \> Author \> : \> Leslie da Silva \\
    \> Website \> : \> www.pudim.com
\end{tabbing}
We can also have more complex tables with \texttt{tabular}.\\\\
\begin{tabular}{|c|c|c|} %ccc means three centered columns, | means vertical line
    \hline
    \head{Text} & \head{Text} & \head{Text}\\
    \hline
    \verb|\textrm| & \verb|\rmfamily| & \rmfamily Example text\\
    \verb|\textrm| & \verb|\rmfamily| & \rmfamily Example text\\
    \verb|\textrm| & \verb|\rmfamily| & \rmfamily Example text\\
    \hline
\end{tabular}
\newpage
\section{How to create references}
Yeah, this is possible. But whatever.
\newpage
\section{How to write math}
Now we are about to start the good shit.\\
The quadratic equation
\begin{equation}
    \label{quad}
    ax^2 + bx + c = 0
\end{equation}
where \im{a, b} and \im{c} are constants and \im{ a \neq 0}, has two solutions for the variable \im{x}:
\begin{equation}
    \label{root}
    x_{1, 2} = \frac{-b \pm \sqrt{b^2-4ac}}{2a}.
\end{equation}
If the \emph{disciminant} \im{\Delta} with \cm{\Delta = b^2 - 4ac} is zero, then the equation (\ref{quad}) has a double solution:
(\ref{root}) becomes \cm{x = - \frac{b}{2a}.}
\LaTeX\ is full of operators, such as log, cos and so on....
\cm{\lim_{n=1, 2, \ldots} a_n \qquad \max_{x<X} x }
\cm{\mathbb{A} = 1, 2, 3...}
\begin{gather}
    x + y + z = 0\\
    y - z = 1
\end{gather}
You can use the \texttt{gahter} package to deal with a system of equations.\\
You can also go to \url{https://detexify.kirelabs.org/classify.html} to manually draw symbols and get the associated command.\\
If you need to use symbols, you can use the \texttt{siunitx} package, but you need to read some documentation first.\\
In \LaTeX, math written inline is called text style, and the math written outside of if is called display style.\\
So, this is a inline example:
\im{
\int_a^b \! f(x) \, dx = \lim_{\Delta x \rightarrow 0}
\sum_{i=1}^{n} f(x_i) \,\Delta x_i
}\\
And this is a display style example:
\cm{
\int_a^b \! f(x) \, dx = \lim_{\Delta x \rightarrow 0}
\sum_{i=1}^{n} f(x_i) \,\Delta x_i
}
\subsection{Building Math Strucutures}
Lets start with how to create arrays
\cm{
    A = \left(
        \begin{array}{cc}
            a_{11} & a_{12} \\
            a_{21} & a_{22}
        \end{array}
        \right)
}
This is the vanilla way. We can also use the \texttt{amsmath} way of doing things. Note that this packages has a slighty different formating. It is not good to mix them up.
\cm{
    \binom{n}{k} = \frac{n!}{k!(n-k)!}
}
We can also have a \im{\overline{\Delta}} and \im{\underbrace{1 + 1 + \cdots + 1}_n}.
We also are able to have mathy accents \im{\widehat{\Delta}}
\begin{thm}
    This is a new theorem.
\end{thm}
\begin{dfn}
    And this is a new definition.
\end{dfn}
Of course you can write a equation inside if and so on... This is just for personal note taking. Don't make it that much complicated.
\end{document}