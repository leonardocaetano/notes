\documentclass[a4paper, 12pt]{article}
\usepackage[left=0.5in, right=0.5in, top=0.8in, bottom=0.8in]{geometry}

\usepackage{hyperref} % this generates the bookmarks for PDFs

\title{\LaTeX \: introduction}
\author{Me}
\date{Jan 27, 2024}

\addtocontents{toc}{\protect{\pdfbookmark[0]{\contentsname}{toc}}}

\begin{document}
\maketitle
\newpage
\tableofcontents
\newpage
\section{Introduction}
Latex is not word processing, but typesetting. That means you have to provide precise instructions with text.\\
Statement \#1:
50\% of \$100 equals to \$50.\\
More special symbols are \&, \_, \{ and \}.\\
Text can be \emph{emphasized.}\\
Besides from \textit{italics}, words can be \textbf{bold}, \textsl{slanted}, or typeset in \textsc{Small Caps}.\\
Such commands can be \textit{\textbf{nested}}.\\
\emph{See how \emph{emphasizing} looks when nested.} \(1+x^2\) \[1+x^2\] % () for inline math, [] for center math, $$ is old

\end{document}