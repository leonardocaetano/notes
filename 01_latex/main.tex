\documentclass[a4paper, 12pt]{book}
\usepackage[a4paper, left=0.5in, right=0.5in, top=0.8in, bottom=0.8in]{geometry}

\usepackage{hyperref} % this generates the bookmarks for PDFs
\usepackage[english]{babel}
\usepackage{blindtext} % generates filler text
\usepackage{paralist} % for the compactenum and compactlist procedures
% ====== MY MACROS

\newcommand{\TUG}{\TeX\ Users Group}
\newcommand{\kw}[2][\bfseries]{{#1#2}}
\newcommand{\im}[1]{{\(#1\)}}
\newcommand{\cm}[1]{{\[#1\]}} % remember that when you want to do center math you don't need to have \\ on EOL
\newcommand{\head}[1]{\textnormal{\textbf{#1}}}

% ====== MY MACROS

\title{\LaTeX\ introduction}
\author{Me}
\date{Jan 27, 2024}

\addtocontents{toc}{\protect{\pdfbookmark[0]{\contentsname}{toc}}}

\begin{document}
\maketitle
\tableofcontents
\chapter*{Preface}
\blindtext[20]
\chapter{Introduction}
\section{Some basics about \LaTeX}
Latex is not word processing, but typesetting. That means you have to provide precise instructions with text.\\
Statement \#1:
50\% of \$100 equals to \$50.\\
More special symbols are \&, \_, \{ and \}.\\
Text can be \emph{emphasized.}\\
Besides from \textit{italics}, words can be \textbf{bold}, \textsl{slanted}, or typeset in \textsc{Small Caps}.\\
Such commands can be \textit{\textbf{nested}}.\\
\emph{See how \emph{emphasizing} looks when nested.}\\ % \(1+x^2\) \[1+x^2\] % () for inline math, [] for center math, $$ is old
\texttt{This is a monospaced font. Also called typewriter font.}\\
A curly brace means in \LaTeX\ that we about to start a group.\\ 
\tiny We \scriptsize can \footnotesize also \small do \normalsize this.\\
The \TUG\ is a result of a custom macro. Use them!\\
Macros can also receive \kw{arguments}.
\cm{1+2=3}
That makes it easier when you want to do \im{2^3=8} for example.
\newpage
\section{How to write "books"}
\blindtext[20]
And as you can see, we need a \footnote{footnote now. We really do. Remember that \im{3^2 = 9}.}
\newpage
\section{How to write lists}
\begin{enumerate}
    \item C
        \begin{itemize}
            \item C89
            \item C99
            \item C11
            \item C17
            \item C23
        \end{itemize}
    \item C++
    \item Pascal
    \item asm 
    \item Rust
\end{enumerate}
Note that \textit{enumerate} and \textit{itemize} has the same syntax.
\begin{description}
    \item[C89] is the best one.
    \item[C23] is a Rust pysop.
\end{description}
We can also use \textit{compactenum} and \textit{compactitem} if we use the package \textit{paralist}.\\
\begin{compactenum}
    \item C
        \begin{compactitem}
            \item C89
            \item C99
            \item C11
            \item C17
            \item C23
        \end{compactitem}
    \item C++
    \item Pascal
    \item asm 
    \item Rust
\end{compactenum}
\newpage
\section{How to create tables}
This looks like a PITA. It probably is.
\begin{tabbing}
    \emph{Info:} \= Software \= :  \= \LaTeX \\
    \> Author \> : \> Leslie da Silva \\
    \> Website \> : \> www.pudim.com
\end{tabbing}
We can also have more complex tables with \textit{tabular}.\\\\
\begin{tabular}{|c|c|c|} %ccc means three centered columns, | means vertical line
    \hline
    \head{Text} & \head{Text} & \head{Text}\\
    \hline
    \verb|\textrm| & \verb|\rmfamily| & \rmfamily Example text\\
    \verb|\textrm| & \verb|\rmfamily| & \rmfamily Example text\\
    \verb|\textrm| & \verb|\rmfamily| & \rmfamily Example text\\
    \hline
\end{tabular}
\newpage
\section{How to create references}
Yeah, this is possible. But whatever.
\newpage
\section{How to write math}
Now we are about to start the good shit.
\end{document}